\documentclass{article}

\usepackage[utf8]{inputenc}
\usepackage[T1]{fontenc}
\usepackage{lmodern}
\usepackage{graphicx}
\usepackage[frenchb]{babel}
\usepackage[table,xcdraw]{xcolor}
\usepackage{geometry}
\usepackage{listings}
\usepackage{fancyhdr}
\usepackage{lastpage}
\usepackage[boxed,french]{algorithm2e}
\usepackage{float}
\usepackage{hyperref}

\newcommand{\info}{\texttt}
\newcommand{\ue}{\textbf{X5I0040 "Algorithmique et Structures de Données 3"}}
\newcommand{\Affiche}{\textbf{afficher}}

\pagestyle{fancy}
\fancyfoot[C]{\thepage/\pageref{LastPage}}
\renewcommand{\headrulewidth}{0.4pt}
\renewcommand{\footrulewidth}{0.4pt}

\title{X5I0040\\ Projet "Jeu Connexion"}
\author{Sidney \bsc{Falhun} \and Corentin \bsc{Chédotal}}
\date{9 Décembre 2016}

\begin{document}

\maketitle
%\thispagestyle{fancy}

\vspace{4cm} %PENSER A ADAPTER LE SAUT DE LIGNE A LA TABLE DES MATIERES

\tableofcontents

\newpage

\section{Introduction}

    Dans le cadre de l'Unité d'Enseignement \ue nous avons été amené à réaliser le Projet "Jeu Connexion". Il consiste en l'implémentation d'un jeu tout en suivant un cahier des charges particulier. Le jeu consiste en la capture de diverses cases d'un plateau carré par deux joueurs dans le but d'intégrer dans ses zones de contrôle le plus de cases étoilées possible. Nous devions programmer notre implémentation du jeu en Java et en utilisant les structures de données vues en cours de notre choix. Enfin concernant l'interface graphique nous n'avions pas de consignes particulières mais il était précisé de ne pas perdre trop de temps dessus, celle-ci n'étant pas notée.
    
    Dans ce rapport, comme demandé, nous expliciterons le fonctionnement de notre vision du projet. En commençant par sa compilation et son exécution, puis en montrant son implémentation et enfin en donnant des jeux de données types permettant de tester le bon fonctionnement du programme.

\section{Exécution et compilation}

    Ce projet utilise un fichier \info{Makefile} pour faciliter la compilation pour l'Utilisateur. Nous allons donc expliciter ci-après les commandes importantes de ce fichier.

    Pour compiler et exécuter le projet l'Utilisateur n'a qu'à faire \info{make}. Le \info{Makefile} s'occupera du reste. Si l'Utilisateur souhaite uniquement compiler le projet il a à sa disposition la commande \info{make compile}. Enfin, sont implémentées dans le fichier les commandes \info{make clean} et \info{make mrproper} afin de supprimer les fichiers intermédiaires et tout les fichiers résultant de la compilation respectivement.  

\section{Implémentation}

    Dans cette section nous allons rentrer dans le détail de l'implémentation du projet en expliquant notre choix de structures de données parmi toutes celles vues en cours. Puis on indiquera en détail comment nous avons répondu aux questions concernant les méthodes à implémenter obligatoirement.

    \subsection{Structures de données choisies}

        La structure de données vues en cours que nous avons décidé d'utiliser est la \emph{Classe-Union}. Initialement on a donc une Classe (au sens de la \emph{Classe-Union} pas de la programmation orientée objet) par case du plateau de jeu. Puis par les utilisations d'Union (par la méthode du même nom dans \info{Plateau.java}) ce sont les Composantes qui vont être les Classes. De plus afin d'optimiser les appels nous avons implémenter une compression des chemins.
        
        L'implémentation exacte choisie a été celle par tableau de pères. En effet dans ce cas précis il faut remonter souvent de fils aux pères afin de savoir à quelle Composante les Cases appartiennent or l'utilisation d'un arbre est optimisée pour le sens inverse. Pour descendre du père au fils. Nous avons donc fait ce choix et nous avons choisi de faire un tableau de pères à deux dimensions. Le Plateau étant à deux dimensions il paraissait logique de stocker ainsi les données concernant les pères des Cases dans une représentation tabulaire précise du Plateau.
        
        Ainsi on peut facilement ajouter des cases au fur et à mesure du déroulement du jeu au différentes Composantes et fusionner les Composantes entre elles (si elles sont de la bonne couleur bien sur).

    \subsection{Implémentation des méthodes obligatoires}

        Le cahier des charges du projet requérait l'implémentation de dix méthodes spécifiques. Dans cette partie nous allons montrer comme demandé notre façon de les intégrer au programme.
        
            \subsubsection{\info{colorerCase()}}
            
                \begin{algorithm}[H]
                \Entree{String col}
                \Sortie{Un booléen indiquant le succès ou non de l'opération}
                \Res{La case est coloriée de la couleur donnée par col}
                \eSi{col = "blanc"}{
                    Case.col = col\;
                    \Retour{vrai}\;
                }{
                    \Affiche\ "Cette case est déjà en couleur."\;
                    \Retour{faux}\;
                }
                \caption{La méthode \info{colorerCase()}}
                \end{algorithm}

                La méthode \info{colorerCase()} prend une chaîne de caractère représentant une couleur et l'appliquera à la Case à condition que la Case soit blanche. En effet on vérifie si la Case a déja recu une couleur. Si oui il n'est pas possible de la recolorer (empêchant ainsi un joueur de prendre une Case à son adversaire). Enfin la méthode retourne un booléen indiquant la réussite ou non du coloriage à des fin de contrôles.
                
                L'algorithme proposé ici est correct puisqu'on ne fait qu'ajouter une couleur à la Case avec la vérification pour empêcher le "vol" de Cases. De plus dans le pire des cas une Case aura toujours une couleur puisqu'elles sont construites avec la couleur blanche affectée de base.
                
                La complexité de cet algorithme est en $O(1)$.
                
                Il n'est pas possible de faire plus efficace qu'$O(1)$.
                
            \subsubsection{\info{afficheComposante()}}
            
                \begin{algorithm}[H]
                \Entree{Case case1, Case case2, String col}
                \Res{Affiche la Composante formée par case1 et case2}
                entier limite = Plateau.longueur - 1\;
                \Si{limite > case2.x OU limite > case2.Y}{
                    \Si{existeCheminCases(case1, case2, col)}{
                        \Affiche\ case2.x + " : " + case2.y\;
                    }
                    case2.x = case2.x + 1\;
                    afficheComposante(case1, case2, col)\;
                    case2.y = case2.y + 1\;
                    afficheComposante(case1, case2, col)\;
                }
                \caption{La méthode \info{afficheComposante()}}
                \end{algorithm}
                
                La méthode \info{afficheComposante()} prend deux Cases et va afficher l'éventuelle Composante les liant. Pour ce faire on vérifie d'abord que les Cases données appartiennent bien au Tableau. Ensuite on teste s'il existe bien un chemin reliant les deux cases entre elles. Si c'est le cas alors on affiche la Composante les reliant. De plus une fois ceci fait on va essayer d'afficher les autres Composantes potentiellement formées par les Cases adjacentes à la première Case.
                
                %PARTIE SUR LA CORRECTION DE L'ALGO, LÀ JE SECHE UN PEU (Corentin)
                
                La complexité de cet algorithme est $O(Plateau.longueur)$. En effet il est executé un nombre de fois dépendant de l'emplacement des Cases sur le Plateau mais aussi et surtout de la longueur de celui-ci. Plus le Plateau sera long plus la complexité temporelle de \info{afficheComposante()} augmentera, de façon linéaire.
                
                %PARTIE SUR L'EFFICACITE DE L'ALGO, LÀ JE SECHE UN PEU (Corentin)
                
            \subsubsection{\info{existeCheminCases()}}
            
                \begin{algorithm}[H]
                \Entree{Case case1, Case case2}
                \Sortie{Un booléen indiquant si case1 et case2 ont le même père}
                compressionChemin(case1.x, case1.y)\;
                compressionChemin(case2.x, case2.y)\;
                \Retour{tableauPeres[case1.x][case1.y] = tableauPeres[case2.x][case2.y]}
                \caption{La méthode \info{rechercheMemePere()} utilisée par \info{existeCheminCases()}}
                \end{algorithm}
                
                \begin{algorithm}[H]
                \Entree{Case case1, Case case2, String col}
                \Sortie{Un booléen indiquant si case1 et case2 bien un chemin de la même couleur les reliant}
                \Retour{rechercheMemePere(case1, case2) ET case1.col = case2.col}
                \caption{La méthode \info{existeCheminCases()} en elle même}
                \end{algorithm}
                
                La méthode \info{existeCheminCases()} recherche d'abord si les deux Case ont le même père dans le tableau de pères. On vérifie donc leur  appartenance à la même Composante. Se faisant on en profite pour appliquer la compression de chemin afin d'optimiser les futures recherches dans celui-ci. Une fois que l'on a confirmation que les deux Cases ont le même père il suffit de s'assurer que celles-ci sont bien de la même couleur. Si c'est le cas alors il existe bien un chemin d'une couleur entre les deux Cases.
                
                L'appartenance à une même Composante est donnée par le tableau de pères. Cette appartenance montre qu'il y a bien un chemin reliant nos deux Cases. La vérification de la couleur portée par celles-ci est une dernière étape afin de s'assurer que l'on respecte bien la partie "de la même couleur" de la consigne. Ainsi l'algorithme est donc correct.
                
                La complexité de celui-ci est en $O(n)$ où $n$ représente le nombre de fois qu'il est nécessaire de faire des appels récursifs dans \info{compressionChemin()} pour faire la compression des chemins de tableau de pères. En effet, par le biais de \info{rechercheMemePere()} on fait appel à \info{compressionChemin()} qui est en $O(n)$ donc n'ayant pas d'autres méthodes appelées et/ou boucle intervenant dans \info{rechercheMemePere()} et dans \info{existeCheminCases()} on a bien aussi $O(n)$ pour cette dernière.
                
                L'algorithme est le plus efficace possible, incorporant la compression de chemin. En effet il serait possible d'avoir ici un algorithme plus efficace sans implémenter la compression de chemin. Cependant on perdrait en réalité sur le long terme puisque les recherches de pères seraient rendues plus longues et moins efficaces. C'est donc un parti pris qui sur la longueur se révélera efficace.
                
            \subsubsection{\info{relierCasesMin()}}
            
                TODO
                
            \subsubsection{\info{getNbEtoiles()}}
                
                \begin{algorithm}[H]
                \Entree{entier x, entier y, String col}
                \Sortie{Le nombre d'étoiles contenues dans la Composante de la Case aux coordonnées [x][y]}
                \Retour{tableauPeres[x][y].nbEtoile()}
                \caption{La méthode \info{getNbEtoiles()}}
                \end{algorithm}
                
                La méthode \info{getNbEtoiles()} va chercher dans le tableau de pères le pères de la Case au coordonnées [x][y] et va retourner la variable du nombre d'étoiles que le père contient. En effet à chaque Union c'est le père qui récupère les étoiles de ses nouveaux fils.
                
                La correction de cette méthode est assez triviale. En effet on ne fait qu'appel à la variable donnant le nombre d'étoiles que contient une Composante. Celle-ci se trouve dans le père de chaque Composante et c'est donc pourquoi on va d'abord chercher le père de la Case aux coordonnées données en entrée.
                
                La complexité de cette méthode est en $O(1)$. En effet on ne fait qu'accéder à un emplacement spécifique dans un tableau et on retourne simplement une variable.
                
                Il n'est pas possible de faire plus efficace qu'$O(1)$.

    \subsection{\emph{optionnel} Méthodes d'évaluation}

\section{Jeux de données}

\section{Conclusion}

\end{document}
